\documentclass[12pt]{article}
\usepackage{amsfonts}
\usepackage{amsmath}
\usepackage{bm}
\usepackage{bbm}
\usepackage{graphicx}
\usepackage{geometry}[margin=1in]
\usepackage{subcaption}
\usepackage{array}
\usepackage{multirow}
\usepackage{hyperref}

% for algorithms
\usepackage[linesnumbered,ruled]{algorithm2e}

\usepackage{natbib}
\usepackage{multicol}
\bibliographystyle{plainnat}
% set path for bibliography

\title{Regionally Additive Models: Explainable-by-design models minimizing feature interactions}

\newcommand{\dfdx}{\frac{\partial f}{\partial x_s}}
\newcommand{\Rd}{\mathbb{R}^d}
\newcommand{\xb}{\mathbf{x}}
\newcommand{\xc}{\mathbf{x_c}}
\newcommand{\xcc}{\mathbf{x_{/s}}}
\newcommand{\fxc}{f^{(\xc)}}
\newcommand{\fxs}{f^{(x_s)}}
\newcommand{\Xcal}{\mathcal{X}}
\newcommand{\Ycal}{\mathcal{Y}}
\newcommand{\when}[1]{\mathbbm{1}_{#1}}

\author{Vasilis Gkolemis}


\begin{document}
    \def\MakeUppercaseUnsupportedInPdfStrings{\scshape}
\maketitle

\begin{abstract}
Generalized Additive Models (GAMs) are widely utilized explainable-by-design models in various applications.
However, their assumption of independence among features can lead to suboptimal performance when violated.
To overcome this limitation, we introduce Regionally Additive Models (RAMs), a novel class of explainable-by-design models.
RAMs mitigate the issue by identifying subregions in the feature space where interactions are minimized and fitting multiple GAMs accordingly.
The RAM framework consists of three steps.
Firstly, we train a black-box model.
Secondly, using Regional Effect Plots, we identify subregions where the black-box model exhibits near-local additivity.
In these subregions, the effect of each feature on the target is independent of the values of other features.
Lastly, we fit a GAM specifically for each identified subregion.
We validate the effectiveness of RAMs through experiments on both a synthetic example and real-world datasets.
The results confirm that RAMs offer improved expressiveness compared to GAMs while maintaining interpretability.
\end{abstract}

\section{Introduction}

% Paragraph for Motivating about Regionally Additive Models
Generalized Additive Models (GAMs)~\citep{hastie1987generalized} are a popular class of explainable by design (x-by-design) models.
Their popularity stems from their seamless interpretability; since they are a linear (additive) combination of univariate functions,
\(f(\xb) = c + \sum_{s=1}^D f_s(x_s)\), each individual univariate function (component) can be readily visualized and comprehended in isolation.
However, GAM's main limitation is that they cannot express interactions between features.
To mitigation this limitation, some approaches~\citep{lou2013accurate} extend them enabling pairwise interactions,
i.e., \(f(\xb) = c + \sum_{s=1}^D f_s(x_s) + \sum_{s=1}^D \sum_{c \neq s} f_{sc}(x_s, x_c)\).
Pairwise interactions can also be visualized and understood in isolation, so these models also maintain their x-by-design nature.
Unfortunately, this does not hold for any interaction that involves more than two features, thus, the expressiveness of GAMs is limited to capturing up to two-feature interactions.

% Paragraph for Motivating about Regionally Additive Models
To overcome this limitation, we propose Regionally Additive Models (RAMs), a novel class of x-by-design models,
that fits multiple GAMs to subregions of the feature space where interactions are minimized.
Our approach consists of a three-step pipeline.
First, we fit a black-box model to capture all high-order interactions.
Second, we identify the subregions where the black-box model is nearly locally additive.
Finally, we train a GAM specifically for each identified subregion.

\section{Background and motivation}
\label{sec:motivation}

Consider the black-box function \(f(\xb) = 8x_2\when{x_1 > 0}\when{x_3=0}\)
with \(x_1, x_2 \sim \mathcal{U}(-1,1)\) and \(x_3 \sim Bernoulli(0,1)\).
Although very simple, a GAM would fail to learn this mapping due to
the existence of the three-features interaction term $8x_2\when{x_1 > 0}\when{x_3=0}$.
As we see in Figure~\ref{subfig:global_gam}, a GAM misleadingly learns that $\hat{f}(\xb) \approx 2x_2$,
because in $\frac{1}{4}$ of the cases ($x_1 > 0 \text{ and } x_3 = 0$) the impact of $x_2$ to the output is $8x_2$,
and in the rest $\frac{3}{4}$ of the cases the impact of $x_2$ to the output is $0$.
However, if splitting the input space in two subregions we observe that \(f\) is additive in each one (regionally additive):
%
\begin{equation}
    \label{eq:regionally_additive}
    f(\xb) = \begin{cases} 8x_2 & \text{if } x_1 > 0 \text{ and } x_3 = 1 \\ 0 & \text{otherwise} \end{cases}
\end{equation}
%
Therefore, if we knew the appropriate subregions,
namely, \(\mathcal{R}_{21} = \{x_1 > 0 \text{ and } x_3 = 0\}\)
and  \(\mathcal{R}_{22} = \{x_1 \leq 0 \text{ or } x_3 = 1\}\),
we could split the impact of $x_2$ appropriately and fit the following model to the data:

\begin{equation}
    \label{eq:regional_gam}
    f^{\mathtt{RAM}}(\xb) = f_1(x_1) + f_{21}(x_2) \when{(x_1, x_3) \in \mathcal{R}_{21}} + f_{22}(x_2) \when{(x_1, x_3) \in \mathcal{R}_{22}} + f_3(x_3)
\end{equation}
%
Equation~\eqref{eq:regional_gam} represents a Regionally Additive Model (RAM), which is simply a GAM fitted on each subregion of the feature space.
Importantly, RAM's enhanced expressiveness does not come at the expense of interpretability.
As we observe in Figures~\ref{subfig:regional_gam_1} and~\ref{subfig:regional_gam_2}, we can still visualize and comprehend each univariate function in isolation, exactly as we would do with a GAM,
with the only difference being that we have to consider the subregions where each univariate function is active,
The key challenge of RAMs is to appropriately identify the subregions where the black-box function is (close to) regionally additive.
For this purpose, as we will see in Section~\ref{subsec:regional_effect_methods}, we propose a novel algorithm that is based on the idea of
regional effect plots.

\begin{figure}[htbp]
    \centering
    \begin{subfigure}{0.32\textwidth}
        \centering
        \includegraphics[width=\textwidth]{figures/global_GAM}
        \caption{\(f_2(x_2)\)}
        \label{subfig:global_gam}
    \end{subfigure}
    \begin{subfigure}{0.32\textwidth}
        \centering
        \includegraphics[width=\textwidth]{figures/regional_gam_subreg_1}
        \caption{\(f_2(x_2) \when{x_1 > 0 \text{ and } x_3 = 1}\)}
        \label{subfig:regional_gam_1}
    \end{subfigure}
    \begin{subfigure}{0.32\textwidth}
        \centering
        \includegraphics[width=\textwidth]{figures/regional_gam_subreg_2}
        \caption{\(f_2(x_2) \when{x_1 \leq 0 \text{ or } x_3 \neq 1}\)}
        \label{subfig:regional_gam_2}
    \end{subfigure}
    \caption{The left image showcases the global GAM which erroneously learns an approximation of \(f(\xb) \approx 2x_2\).
    In contrast, the middle and right images demonstrate the RAM's ability to identify two distinct subregions where \(f\) exhibits regional additivity.
    By fitting a GAM to each subregion, the RAM accurately captures the true function $f$ while retaining interpretability.}
    \label{fig:ram_example}
\end{figure}


\section{The RAM framework}

The RAM framework consists of a three-step pipeline; (a) fit a black-box model (Section~\ref{subsec:fit_black_box}),
(b) identify subregions with minimal interactions (Section~\ref{subsec:regional_effect_methods}) and
(c) fit a GAM to each subregion (Section~\ref{subsec:fitting_gams}).
Throughout this section, we will use the following notation.
Let \(\Xcal \in \Rd\) be the \(d\)-dimensional feature space, \(\Ycal\) the target space and \(f(\cdot) : \Xcal \rightarrow \Ycal\) the black-box function.
We use index \(s \in \{1, \ldots, d\}\) for the feature of interest and \(/s = \{1, \ldots, D\} - s\) for the rest.
For convenience, we use \((x_s, \xcc)\) to refer to \((x_1, \cdots , x_s, \cdots, x_D)\) and, equivalently, \((X_s, X_{/s})\) instead of \((X_1, \cdots , X_s, \cdots, X_D)\) when we refer to random variables.
The training set \(\mathcal{D} = \{(\xb^i, y^i)\}_{i=1}^N\) is sampled
i.i.d.\ from the distribution \(\mathbb{P}_{X,Y}\).
%Finally, \(f^{\mathtt{<method>}}(x_s)\) denotes how \(\mathtt{<method>}\)
%defines the feature effect and \(\hat{f}^{\mathtt{<method>}}(x_s)\)
%how it estimates it from the training set.

\subsection{First step: Fit a black-box function}
\label{subsec:fit_black_box}

In the initial step of the pipeline, we fit a black-box function \(f(\cdot)\) to the training set
\(\mathcal{D} = \{(\xb^i, y^i)\}_{i=1}^N\) to accurately learn the underlying mapping \(f(\cdot) : \Xcal \rightarrow \Ycal\).
While any black-box function can theoretically be employed in this stage,
for utilizing the DALE approximation, as we will show in the next step,
it is necessary to select a differentiable function.
Recent advancements have demonstrated that differentiable Deep Learning models,
specifically designed for tabular data~\citep{arik2021tabnet}, are capable of achieving state-of-the-art performance,
making them a suitable choice for this step.
% Driver, teacher, guide

\subsection{Second step: Find subregions}
\label{subsec:regional_effect_methods}

In this step, we use regional effect methods~\citep{herbinger2023decomposing, herbinger2022repid}
to identify the regions where the black-box function is (close to) regionally additive.
% Describe the goal of the regional effect methods
Regional effect methods yield for each individual feature \(s\), a set of \(T_s\) non-overlapping regions,
denoted as \(\{\mathcal{R}_{st}\}_{t=1}^{T_s}\) where \(\mathcal{R}_{st} \subseteq \Xcal_{/s}\).
Note that, the number of non-overlapping regions can be different for each feature ($T_s$),
the regions \(\{\mathcal{R}_{st}\}_{t=1}^{T_s}\) are disjoint
and their union covers the entire feature space \(\Xcal_{/s}\).
The primary objective is to identify regions in which the impact of the \(s\)-th feature on the output is
\textit{relatively independent} of the values of the other features \(\xcc\).
To better grasp this objective, if we decompose the impact of the \(s\)-th feature on the output $y$ into two terms:
\(f_s(x_s, \xcc) = f_{s,ind}(x_s) + f_{s, int}(x_s, \xcc)\),
where \(f_{s,ind}(\cdot)\) represents the independent effect
and \(f_{s, int}(\cdot)\) represents the interaction effect,
the objective is to identify regions \(\{\mathcal{R}_{st}\}_{t=1}^{T_s}\) such that the interaction effect is minimized.
Regionally Additive Models (RAM) formulate the mapping \(\mathcal{X} \rightarrow \mathcal{Y}\) as:

\begin{equation}
\label{eq:ram_formulation}
f^{\mathtt{RAM}}(\xb) = c + \sum_{s=1}^D \sum_{t=1}^{T_s} f_{st}(x_s) \when{\xcc \in \mathcal{R}_{st}}, \quad \xb \in \Xcal
\end{equation}
%
In the above formulation, \(f_{st}(\cdot)\) is the component of the \(s\)-th feature which is active on the \(t\)-th region.
RAM can be viewed as a GAM with \(T_s\) components per feature where each component is applied to a specific region \(\mathcal{R}_{st}\).
To facilitate this interpretation, we can define an enhanced feature space \(\Xcal^\mathtt{RAM}\) defined as:

\begin{equation}
\label{eq:ram_feature_space}
\begin{aligned}
\Xcal^{\mathtt{RAM}} &= \{x_{st} | s \in \{1, \ldots, D\}, t \in \{1, \ldots, T_s\}\} \\
x_{sk} &= \begin{cases}
x_s, & \text{if } \xcc \in \mathcal{R}_{sk} \\
0, & \text{otherwise}
\end{cases}
\end{aligned}
\end{equation}
%
and then define RAM as a typical GAM on the extended feature space \(\Xcal^{\mathtt{RAM}}\):

\begin{equation}
\label{eq:ram_formulation2}
f^{\mathtt{RAM}}(\xb) = c + \sum_{s,t} f_{st}(x_{st}) \quad \xb \in \Xcal^{\mathtt{RAM}}
\end{equation}
%
Equations~\ref{eq:ram_formulation} and~\ref{eq:ram_formulation2} are equivalent.
To gain a better understanding of the latter formulation, consider the toy example described in Section~\ref{sec:motivation}.
To minimize the impact of feature interactions, we need to divide feature \(x_2\) into two subregions,
\(\mathcal{R}_{21} = \{x_1 > 0 \text{ and } x_3 = 1\}\) and \(\mathcal{R}_{22} = \{x_1 \leq 0 \text{ or } x_3 = 0\}\).
This division results in an augmented feature space \(\Xcal^{\mathtt{RAM}} = (x_1, x_{21}, x_{22}, x_3)\)
and a RAM formulation of the form: \(f^{\mathtt{RAM}}(\xb) = f_1(x_1) + f_{21}(x_{21}) + f_{22}(x_{22}) + f_3(x_3)\).


\subsubsection{Proposed Approach}

To identify the regions of the input space where the impact of feature interactions is reduced,
we have developed a regional effect method influenced by the research conducted by
\citet{herbinger2023decomposing} and \citet{gkolemis2023dale}.
\citet{herbinger2023decomposing} introduced a versatile framework for detecting such regions,
where one of the proposed methods is the Accumulated Local Effects~\citep{apley2020visualizing}.
We have adopted their approach with two notable modifications.
First, instead of using the ALE plot, we employ the Differential ALE (DALE) method introduced by \citet{gkolemis2023dale},
which provides considerable computational advantages when the underlying black-box function is differentiable.
Second, we utilize variable-size bins, instead of the fixed-size ones in DALE, because the result in a more accurate approximation.

\paragraph{DALE formulation}

DALE gets as input the black-box function \(f(\cdot)\)
and the dataset \(\mathcal{D} = \{(\xb^i, y^i)\}_{i=1}^N\),
and returns the effect (impact) of a specific feature $s$ on the output $y$:
%
\begin{equation}  \label{eq:DALE_accumulated_mean_est}
  \hat{f}^{\mathtt{DALE}}(x_s) = \Delta x \sum_{k=1}^{k_x} \underbrace{\frac{1}{|\mathcal{S}_k|} \sum_{i:\mathbf{x}^{(i)} \in
    \mathcal{S}_k} \dfdx(\mathbf{x}^i)}_{\hat{\mu}(z_{k-1}, z_k)})
\end{equation}
%
For more details on the DALE method, please refer to the original paper~\citep{gkolemis2023dale}.
In the above equation, \(k_x\) is the index of the bin such that
\(z_{k_x-1} \leq x_s < z_{k_x} \) and \(\mathcal{S}_k\)
is the set of the instances of the \(k\)-th bin, i.e.
\( \mathcal{S}_k = \{ \xb^i : z_{k-1} \leq x^{(i)}_s < z_{k} \} \).
In short, DALE computes the average effect (impact) of the feature \(x_s\) on the output,
by, first, dividing the feature space into $K$ equally-sized bins, i.e., \(z_0, \ldots, z_K\)
second, computing the average effect in each bin \(\hat{\mu}(z_{k-1}, z_k)\) (bin-effect) as the average of the instance-level effects inside the bin,
and, finally, aggregating the bin-level effects.

\paragraph{DALE for feature interactions}

In cases where there are strong interactions between the features,
the instance-level effects inside each bin deviate from the average bin-effect.
We can measure such deviation using the standard deviation of the instance-level effects inside each bin (bin-deviation):

\begin{equation}
  \label{eq:var_bin_approx}
  \hat{\sigma}^2(z_{k-1}, z_k) = \frac{1}{|\mathcal{S}_k| - 1}
\sum_{i:\mathbf{x}^i \in \mathcal{S}_k} \left ( \dfdx(\mathbf{x}^i) -
  \hat{\mu}(z_{k-1}, z_k) \right )^2
\end{equation}
%
and the interaction between the feature \(x_s\) and the rest of the features along the whole \(s\)-th dimension
with the aggregated bin-deviation:
\begin{equation}
  \label{eq:DALE_interaction}
  \mathcal{H}_s = \sqrt{ \sum_{k=1}^{k_x} (z_k - z_{k-1})^2 \hat{\sigma}^2(z_{k-1}, z_k) }
\end{equation}
%
Eq.~\eqref{eq:DALE_interaction} measures the interaction between the feature \(x_s\) and the rest of the features.
It takes values in the range \([0, \infty)\) with zero indicating that \(x_s\) does not interact with the rest of the features,
i.e., the underlying black box function can be written as $f(\xb) = f_s(x_s) + f_{/s}(x_{/s})$.
In all other cases, $\mathcal{H}_s$ is greater than zero and the higher the value, the stronger the interaction.

A final detail, is that in order to have a more robust estimation of the bin-effect and the bin-deviation,
we use variable-size bins instead of the fixed-size ones in DALE.
In particular, we start with a dense fixed-size grid of bins and we iteratively merge the neighboring bins with similar
bin-effect and bin-deviation until all bins have at least $n_{\min}$ instances.
In this way, we can have a more accurate approximation of the bin-effect and the bin-deviation.

\paragraph{Subregions as an optimization problem}

In the same way that we can estimate the feature effect and the feature interactions for the $s$-th feature in the whole input space,
using Eq.~\eqref{eq:DALE_accumulated_mean_est} and Eq.~\eqref{eq:DALE_interaction},
we can also estimate the effect and the interactions in a subregion of the input space \(\mathcal{R}_{st} \subset \mathcal{X}\).
We denote the equivalent regional qunatities as $f^{\mathtt{DALE}}_{\mathcal{R}_{st}}(x_s)$ and $\mathcal{H}_{\mathcal{R}_{st}}$.
$f^{\mathtt{DALE}}_{\mathcal{R}_{st}}(x_s)$ and $\mathcal{H}_{\mathcal{R}_{st}}$ are defined exactly as in
Eq.~\eqref{eq:DALE_accumulated_mean_est} and Eq.~\eqref{eq:DALE_interaction} respectively,
with the only difference that instead of using the whole dataset \(\mathcal{D}\) to compute the regional bin-effect $\hat{\mu}_{\mathcal{R}_{st}}(z_{k-1}, z_k)$
and the regional bin-deviation $\hat{\sigma}_{\mathcal{R}_{st}}^2(z_{k-1}, z_k)$,
we use only the instances that belong to the subregion \(\mathcal{R}_{st}\), i.e. $\xb^i: x_s^i \in \mathcal{S}_k \land x_{s}^i \in \mathcal{R}_{st}$.
Therefore, in order to minimize the interactions of a particular feature $s$ we search for a set of regions \(\{\mathcal{R}_{st}\}_{t=1}^{T_s}\),
that minimize:

\begin{equation}
  \label{eq:optimal_subregions}
  \begin{aligned}
    & \underset{\{\mathcal{R}_{st}\}_{t=1}^{T_s}}{\text{minimize}}
    & & \mathcal{L} = \sum_{t=1}^{T_s} \mathcal{H}_{\mathcal{R}_{st}} \\
    & \text{subject to}
    & & \bigcup_{t=1}^{T} \mathcal{R}_{st} = \mathcal{X} \\
    & & & \mathcal{R}_{st} \cap \mathcal{R}_{s\tau} = \emptyset, \quad \forall t \neq \tau
  \end{aligned}
\end{equation}
%



\paragraph{Proposed solution}

To minimize the objective of Eq.~\eqref{eq:SE}, we have developed a tree-based algorithm based on the approach proposed
by~\citep{herbinger2023decomposing}.
The core of the algorithm is outlined in Algorithm~\ref{alg:subregion_detection}.
We search for subregions independently for each feature $s \in \{1, \ldots, D\}$.
and we select the $T$ optimal splits, in a greedy manner.
To better comprehend the algorithm, let's consider the illustrative example of Section~\ref{sec:motivation}.
For feature $s=2$, the algorithm starts with the first level of the tree, with $/s = \{1, 3\}$ as candidate split-features.
Algorithm~\ref{alg:single_feature_subregion} identifies the optimal first-level split, with the following procedure.
Initially, it determines the candidate split points for each feature.

Since $x_1$ is a continuous feature, are generated as a linearly spaced grid of $P$ points within the range of the feature, i.e. $[-1, 1]$,
where $P$ is a hyperparameter of the algorithm, set to 10 in the experiments.
Consequently, the candidate split points $p \in \{-1, -0.8, -0.6, \ldots, 0.8, 1\}$ define the candidate subregions
$\mathcal{R}_{21} = \{ \mathbb{R}^2 : x_1 \leq p \}$ and $\mathcal{R}_{22} = \{ \mathbb{R}^2 : x_1 > p \}$.
As for $x_3$, being a categorical feature, the candidate split points are its unique values, i.e., $\{0, 1\}$,
and the corresponding subregions are $\mathcal{R}_{21} = \{ \mathbb{R}^2 : x_3 = 0 \}$ and $\mathcal{R}_{22} = \{ \mathbb{R}^2 : x_3 \neq 0 \}$.

For each candidate position, the algorithm splits the initial dataset into two subsets,
$[\mathcal{D}_1, \mathcal{D}_2]$ one for each subregion and computes the level of interaction
as the weighted sum of the interactions of the two subregions, as defined by Algorithm~\ref{alg:get_interaction}.
After the iteration over all points and all candidate positions,
the split point that minimizes the weighted level of interactions is selected.
In the illustrative example, the optimal first-level subregions are
$\mathcal{R}_{21} = \{ \mathbb{R}^2 : x_3 = 0 \}$ and $\mathcal{R}_{22} = \{ \mathbb{R}^2 : x_3 \neq 0 \}$.
The algorithm then proceeds to the second level of the tree, where the only candidate feature is $x_3$.
In this step, the first split is considered fixed, i.e., the algorithm uses a list of two datasets,
and searches in an analogous manner for the optimal second-level split that will be applied on top of the first split,
leading to four subregions.
The algorithm proceeds in a similar manner, until it reaches the maximum depth $T$.

The output of algorithm~\ref{alg:subregion_detection} is a collection of $T$ splits for each feature $s$.
It is important to note that not all splits lead to a significant decrease in the level of interactions.
To address this, we post-process the splits by pruning any splits that do not result in a decrease of the
weighted interaction greater than a threshold value $\epsilon$ (set to $20\%$ drop in the experiments).
The pruned set of splits is denoted as $\{\mathcal{R}_{st}\}_{t=1}^{T_s}$.


\begin{algorithm}
\caption{Detection of Subregions using DALE}
\label{alg:subregion_detection}
\SetAlgoLined
\SetKwInOut{Input}{Input}
\SetKwInOut{Output}{Output}
\BlankLine
\Input{Data matrix $X$, Jacobian matrix $J$, Maximum Depth $T$}
\BlankLine
\Output{Optimal splits}
\BlankLine
Initialize empty arrays for \texttt{loss}, \texttt{position}, and \texttt{feature\_c}
\BlankLine
\For{$s = 1$ to $D$}{
X\_list, J\_list = [X], [J] \tcp*{Init lists}
\For{$t = 1$ to $T$}{
    \tcc{Find best split}
    L, p, c, X\_list, J\_list $\gets$ BestSplit(X\_list, J\_list, s, is\_cat)\;
    \tcc{Store best split}
    loss[s, t], position[s, t], feature\_c[s, t] $\gets$ L, p, c
}
}
\Return{loss, position, feature\_c}
\end{algorithm}

\begin{algorithm}
\caption{BestSplit}
\label{alg:single_feature_subregion}
\SetAlgoLined
\SetKwInOut{Input}{Input}
\SetKwInOut{Output}{Output}

\Input{X\_list, J\_list, s, c, is\_cat}
\Output{BestSplits}

% Initialize variables
\BlankLine
L[c, p] $\gets$ None\; % \Comment{List to store interaction values}
\For{c $s = 1$ to $D$ \textnormal{if} c $\neq$ s}{
    is\_cat $\gets$ IsCategorical(X\_list, c)\;
    positions $\gets$ GetPositions(X\_list, c, is\_cat)\;
    L\_pos $\gets$ []\; % \Comment{List to store interaction values}
    \For{p \textnormal{in} positions}{
        X\_split, J\_split $\gets$ SplitDataset(X\_list, J\_list, c, p, is\_cat)\;
        L $\gets$ GetInteraction(X\_split, J\_split)\;
    }
}
L\_min $\gets$ min(L)\;
c\_min, p\_min = argmin(L)\;
is\_cat $\gets$ IsCategorical(X\_list, c\_min)\;
X\_split, J\_split $\gets$ SplitDataset(X\_list, J\_list, c\_min, p\_min, is\_cat)\;

\Return{L\_min, p\_min c\_min, X\_split, J\_split}
\BlankLine
\end{algorithm}

\begin{algorithm}
\caption{GetInteraction}
\label{alg:get_interaction}
\SetAlgoLined
\SetKwInOut{Input}{Input}
\SetKwInOut{Output}{Output}
\Input{$X\_list$, $J\_list$, $min\_points$}
\Output{weighted average of interaction levels}
N $\gets$ total number of items in $X\_list$\;
W $\gets$ []\;
L $\gets$ []\;
\For{$i$ in len($X\_list$)}{
    X $\gets$ X\_list[i]\;
    J $\gets$ J\_list[i]\;
    \uIf{$|X|$ < min\_points}{
      Append $\infty$ to L\;
    }
    \Else{
        Append $\mathcal{L}(X, J)$ to $L$\;
    }
    Append $\frac{|X|}{N}$ to $W$\;
}
\Return $\text{sum}(L \cdot W)$\;
\end{algorithm}



\paragraph{Computational Complexity}


Algorithm~\ref{alg:subregion_detection} has a computational complexity of $\mathcal{O}(D \cdot T)$
as it iterates over all features and levels of the tree.
Algorithm~\ref{alg:single_feature_subregion} has a computational complexity of $\mathcal{O}(D \cdot P \cdot N)$
as it iterates over all features, query positions, and performs indexing operations on the data (SplitDataset and GetInteraction).
The use of DALE eliminates the need to compute the Jacobian matrix for each split,
resulting in computational complexity independent of the model complexity.
Considering the entire algorithm, the computational complexity is $\mathcal{O}(D \cdot (D-1) \cdot T \cdot P \cdot N)$.
However, in practice, $P$ and $T$ are small numbers.
Therefore, the computational complexity of the proposed method simplifies to $\mathcal{O}(D^2 \cdot N)$,
making it suitable for large datasets, heavy models, and reasonably high-dimensional data.
In practice, for a typical scenario with $N=10 \times 10^3$ samples and $D=20$ features, the proposed method takes less than $10$ seconds to run.

\subsection{Third step: Fit a GAM in each subregion}
\label{subsec:fitting_gams}

Once the subregions are detected, any Generalized Additive Model (GAM) family can be fitted to the augmented input space $\mathcal{X}^{\mathtt{RAM}}$.
Recently, several methods have been proposed to extend GAMs and enhance their expressiveness.
These methods can be categorized into two main research directions.
The first direction targets on representing the main components of a GAM $\{ f_i(x_i) \}$ using novel models.
For example,~\citep{agarwal2021neural} introduced an approach that employs an end-to-end neural network to learn the main components.
The second direction aims to extend GAMs to model feature interactions.
Examples of such extensions include Explainable Boosting Machines (EBMs)~\citep{lou2013accurate} or
Node-GAMs~\citep{chang2021node}.
These models are generalized additive models that incorporate pairwise interaction terms.
It is worth noticing, that the RAM framework and can be used on top of both these research directions
to further enhance the expressiveness of the models while maintaining their interpretability.
In our experiments, we use the Explainable Boosting Machines (EBMs).


\section{Experiments}

We evaluate the proposed approach on two typical tabular datasets, namely the Bike-Sharing Dataset~\citep{misc_bike_sharing_dataset_275}
and the California Housing Dataset~\citep{pace1997sparse}.


\begin{table}[htbp]
  \centering
  \caption{The table compares RMSE values of different models on two datasets: Bike-Sharing and California Housing.
  The models include DNN, GAM, RAM, GAM², and RAM² (representing 2nd order interactions).
  Lower values indicate better performance.
  RAM consistently outperforms GAM and approaches DNN performance.}
  \label{tab:sample}
  \begin{tabular}{l|c|cccc}
      \hline
      & \textbf{Black-box} & \multicolumn{4}{c}{\textbf{x-by-design}} \\
      \hline
      \hline
      & all orders & \multicolumn{2}{c}{1\textsuperscript{st} order} & \multicolumn{2}{c}{2\textsuperscript{nd} order} \\
      \hline
      \hline
      & \textbf{DNN} & \textbf{GAM} & \textbf{RAM} & \textbf{GAM}$^2$ & \textbf{RAM}$^2$ \\
      \hline
      Bike Sharing      & 0.254 & 0.549 & 0.430 & 0.298 & 0.278 \\
      \hline
      California Housing      & 0.369 & 0.600 & 0.535 & 0.554 & 0.514 \\
  \end{tabular}
\end{table}

\paragraph{Bike-Sharing Dataset}

The Bike-Sharing dataset contains the hourly bike rentals in the state of Washington DC over the period 2011 and 2012.
The dataset contains a total of 14 features, out of which 11 are selected as relevant for the purpose of prediction.
The majority of these features involve measurements related to environmental conditions,
such as $X_{\mathtt{month}}$, $X_{\mathtt{hour}}$, $X_{\mathtt{temperature}}$, $X_{\mathtt{humidity}}$ and $X_{\mathtt{windspeed}}$.
Additionally, certain features provide information about the type of day, for example, whether it is a working day ($X_{\mathtt{workingday}}$) or not.
The target value \( Y_{\mathtt{count}}\) is the bike rentals per hour, which has mean value
\(\mu_{\mathtt{count}} = 189\) and standard deviation \(\sigma_{\mathtt{count}} = 181\).

% Training - Evaluation - Region Extraction
As a black-box model, we train for \(60\) epochs a fully-connected Neural Network with 6 hidden layers, using the Adam optimizer with a learning rate of $0.001$.
The model attains a root mean squared error of \( 0.25 \cdot 181 \approx 45\) counts on the test set.
Subsequently, we extract the subregions, searching for splits up to a maximum spliting depth of \(T=3\).
Following the postprocessing step, we find that the only split that substantially reduces the level of interactions within the subregions is based on the feature
$X_{\mathtt{hour}}$. This feature is divided into two subgroups: $X_{\mathtt{hour}} | \when{X_{\mathtt{workingday}} \neq 1}$ and $X_{\mathtt{hour}} \when{X_{\mathtt{workingday} = 1}}$.

Figure~\ref{fig:bike_sharing} clearly illustrates that the impact of the hour of the day on bike rentals varies
significantly depending on whether it is a working day or a non-working day.
Specifically, during working days, there is higher demand for bike rentals in the morning and afternoon hours,
which aligns with the typical commuting times (\ref{subfig:bike_rentals_regional_gam_1}).
On the other hand, during non-working days, bike rentals peak in the afternoon as individuals engage in
leisure activities (\ref{subfig:bike_rentals_regional_gam_2}).
The proposed RAM method effectively captures and detects this interaction by establishing two distinct subregions,
each corresponding to working days and non-working days, respectively.
Subsequently, the EBM that is fitted to each subregion, successfully learns these patterns,
achieving a root mean squared error of approximately \( 0.43 \cdot 181 \approx 77\) counts on the test set.
It is noteworthy that RAM not only preserves the interpretability of the model,
but it also enhances the interpretation of the underlying modeling process.
By identifying and highlighting the interaction between the hour of the day and the day type,
RAM provides valuable insights into the relationship between these variables and their influence on bike rentals.
In contrast, the GAM model~\ref{subfig:bike_rentals_gam} is not able to capture this interaction and
achieves a root mean squared error of \( 0.55 \cdot 181 \approx 100\) counts on the test set.
Finally, in table~\ref{tab:sample}, we also observe that the RAM$^2$, i.e., RAM with second-order interactions,
outperforms the equivalent GAM$^2$ model in terms of predictive performance.

\begin{figure}[htbp]
    \centering
    \begin{subfigure}{0.32\textwidth}
        \centering
        \includegraphics[width=\textwidth]{figures/bike_rentals_gam}
        \caption{\(f(X_{\mathtt{hour}})\)}
        \label{subfig:bike_rentals_gam}
    \end{subfigure}
    \begin{subfigure}{0.32\textwidth}
        \centering
        \includegraphics[width=\textwidth]{figures/bike_rentals_ram_1}
        \caption{\(f(X_{\mathtt{hour}}) \when{X_{\mathtt{workingday}} \neq 1}\)}
        \label{subfig:bike_rentals_regional_gam_1}
    \end{subfigure}
    \begin{subfigure}{0.32\textwidth}
        \centering
        \includegraphics[width=\textwidth]{figures/bike_rentals_ram_2}
        \caption{\(f(X_{\mathtt{hour}}) \when{X_{\mathtt{workingday}} = 1}\)}
        \label{subfig:bike_rentals_regional_gam_2}
    \end{subfigure}
    \caption{Comparison of different models' predictions for bike rentals based on the hour of the day.
    Subfigure (a) depicts the generalized additive model (GAM),
        while subfigures (b) and (c) illustrate the RAM model's predictions for different day types: non-working days
        \(f(X_{\mathtt{hour}}) \when{X_{\mathtt{workingday}} \neq 1}\) and
        working days \(f(X_{\mathtt{hour}}) \when{X_{\mathtt{workingday}} = 1}\), respectively.
        The RAM model successfully captures the interaction between the hour of the day and the day type,
        leading to improved predictions and enhanced interpretability.}
    \label{fig:bike_sharing}
\end{figure}

\paragraph{California Housing Dataset}


The California Housing dataset consists of approximately $20,000$ of housing blocks situated in California.
Each housing block is described by eight numerical features, namely,
$X_{\mathtt{lat}}$, $X_{\mathtt{long}}$, $X_{\mathtt{median\_age}}$, $X_{\mathtt{total\_rooms}}$, $X_{\mathtt{total\_bedrooms}}$, $X_{\mathtt{population}}$, $X_{\mathtt{households}}$, and $X_{\mathtt{median\_income}}$.
The target variable, $Y_{\mathtt{value}}$, is the median house value in dollars for each block.
The target value ranges in the interval \([15, 500] \cdot 10^3\), with a mean value of
\(\mu_Y \approx 201 \cdot 10^3 \) and a standard deviation of \(\sigma_Y \approx 110 \cdot 10^3\).

As a black-box model, we train for $45$ epochs a fully-connected Neural Network with 6 hidden layers,
using the Adam optimizer with a learning rate of $0.001$.
The model achieves a root mean square error (RMSE) of about \(40\)K dollars on the test set.
Subsequently, we perform subregion extraction by searching for splits up to a maximum depth of \(T=3\).
After the postprocessing step, we discover that several splits significantly reduce the level of interactions,
resulting in an expanded input space consisting of \(16\) features, as we show in table~\ref{tab:california_housing_subregions}.
Out of them, we randomly select and illustrate in Figure~\ref{fig:california_housing} the effect of the feature $X_{\mathtt{long}}$.
As we observe, for the house blocks located in the southern part of California ($X_{\mathtt{lat}} \leq 34.9$),
the house value decreases in an almost linear fashion as we move eastward ($X_{\mathtt{long}}$ increases).
In contrast, for the house blocks located in the northern part of California ($X_{\mathtt{lat}} > 34.9$),
the house value performs a rapid (non-linear) decrease as we move eastward ($X_{\mathtt{long}}$ increases).
We also observe that although the EBM fitted to each subregion captures the general trend,
it does not align perfectly with the regional effect.
As in the Bike-Sharing Example, the RMSE of the RAM model, i.e. \( 0.53 \cdot 110 \cdot 10^3 \approx 58.3 \cdot 10^3\) dollars on the test set,
is lower than the one of the GAM model, i.e.\( 0.6 \cdot 110 \cdot 10^3 \approx 66 \cdot 10^3\) dollars.
These results indicate that the RAM model provides superior predictions compared to the GAM model.
The same conclusion holds is when comparing the RAM$^2$ and the GAM$^2$ models.
\begin{table}[htbp]
  \centering
  \caption{California Housing: Subregions Detected by RAM}
  \label{tab:california_housing_subregions}
  \begin{tabular}{c|c}
      Feature & Subregions \\
      \hline
      \multirow{2}{*}{$X_{\mathtt{long}}$} & $X_{\mathtt{long}} \when{X_{\mathtt{lat}} \leq 34.9}$ \\
      & $X_{\mathtt{long}} \when{X_{\mathtt{lat}} > 34.9}$ \\
      \hline
      \multirow{2}{*}{$X_{\mathtt{lat}}$} & $X_{\mathtt{lat}} \when{X_{\mathtt{long}} \leq -120.31}$ \\
      & $X_{\mathtt{lat}} \when{X_{\mathtt{long}} > -120.31}$ \\
      \hline
      \multirow{2}{*}{$X_{\mathtt{total\_rooms}}$} & $X_{\mathtt{total\_rooms}} \when{X_{\mathtt{total\_bedrooms}} \leq 449.37}$ \\
        & $X_{\mathtt{total\_rooms}} \when{X_{\mathtt{total\_bedrooms}} > 449.37}$ \\
      \hline
      \multirow{4}{*}{$X_{\mathtt{total\_bedrooms}}$} & $X_{\mathtt{total\_bedrooms}} \when{X_{\mathtt{households}} \leq 411} \when{X_{\mathtt{total\_bedrooms}} \leq 647}$ \\
        & $X_{\mathtt{total\_bedrooms}} \when{X_{\mathtt{households}} \leq 411} \when{X_{\mathtt{total\_bedrooms}} > 647}$ \\
        & $X_{\mathtt{total\_bedrooms}} \when{X_{\mathtt{households}} > 411} \when{X_{\mathtt{total\_bedrooms}} \leq 647}$ \\
        & $X_{\mathtt{total\_bedrooms}} \when{X_{\mathtt{households}} > 411} \when{X_{\mathtt{total\_bedrooms}} > 647}$ \\
      \hline
      \multirow{2}{*}{$X_{\mathtt{population}}$} & $X_{\mathtt{population}} \when{X_{\mathtt{households}} \leq 411.5}$ \\
      & $X_{\mathtt{population}} \when{X_{\mathtt{households}} > 411.5}$ \\
      \hline
      \multirow{2}{*}{$X_{\mathtt{households}}$} & $X_{\mathtt{households}} \when{X_{\mathtt{total\_bedrooms}} \leq 630.57}$ \\
        & $X_{\mathtt{households}} \when{X_{\mathtt{total\_bedrooms}} > 630.57}$ \\
  \end{tabular}
\end{table}


\begin{figure}[htbp]
    \centering
    \begin{subfigure}{0.32\textwidth}
        \centering
        \includegraphics[width=\textwidth]{figures/california_gam}
        \caption{\(f(X_{\mathtt{long}})\)}
        \label{subfig:california_gam}
    \end{subfigure}
    \begin{subfigure}{0.32\textwidth}
        \centering
        \includegraphics[width=\textwidth]{figures/california_ram_1}
        \caption{\(f(X_{\mathtt{long}}) \when{X_{\mathtt{lat}} \leq 34.89}\)}
        \label{subfig:california_ram_1}
    \end{subfigure}
    \begin{subfigure}{0.32\textwidth}
        \centering
        \includegraphics[width=\textwidth]{figures/california_ram_2}
        \caption{\(f(X_{\mathtt{long}}) \when{X_{\mathtt{lat}} > 34.89}\)}
        \label{subfig:california_ram_2}
    \end{subfigure}
    \caption{Comparison of different predictions for housing prices in California based on the longitude.
    Subfigure (a) showcases the generalized additive model (GAM),
        while subfigures (b) and (c) demonstrate the RAM components for different latitude ranges:
        \(f(X_{\mathtt{long}}) \when{X_{\mathtt{lat}} \leq 34.89}\) and
        \(f(X_{\mathtt{long}}) \when{X_{\mathtt{lat}} > 34.89}\), respectively.
        We observe, that although the EBM model is able to capture the overall trend in the data,
        it also exhibits a large amount of variance.}
    \label{fig:california_housing}
\end{figure}

\section{Conclusion and Future Work}

In this paper we have introduced the Regional Additive Models (RAM) framework, a novel approach for learning accurate
x-by-design models from data.
RAMs operate by decomposing the data into subregions, where the relationship between the target variable and the
features exhibits an approximately additive nature.
Subsequently, Generalized Additive Models (GAMs) are fitted to each subregion and combined to create the final model.
Our experiments on two standard regression datasets have shown promising results, indicating that RAMs can provide more accurate predictions compared to GAMs while maintaining the same level of interpretability.

Nevertheless, there are still several unresolved questions that require attention and further experimentation.
Firstly, it is essential to systematically evaluate the performance of RAMs on a larger set of datasets to ensure that the observed improvements are not specific to particular datasets.
Secondly, we need to explore different approaches for each step of the RAM framework.
For the initial step, we should experiment with various black-box models.
Regarding the subregion detection step, we can explore alternative clustering algorithms.
Finally, in the last step, we should investigate different types of GAM models to fit within each subregion.

Another important area of investigation involves exploring the impact of second-order effects within the RAM framework.
While our experimenation demonstrated that even with the current subregion detection, RAM$^2$s outperform GAM$^2$s,
it may be the case, that for second-order models the optimal subregions are not necessarily those that maximize the additive effect of individual features,
but rather those that maximize the additive effect of feature pairs.

%\section{Appendix}
%
%\subsection{Regional DALE formulation}
%\label{sec:regional_dale}
%
%Below we provide the regional DALE formulation for the bin-effect, the bin-deviation and the aggregated deviation on a subregion \(\mathcal{R}_{st}\):
%
%\begin{equation}
%    \label{eq:bin_effect_subregion}
%    \hat{\mu}_{\mathcal{R}_{st}}(z_{k-1}, z_k) = \frac{1}{|\mathcal{S}_k \cap \mathcal{R}_{st}|} \sum_{i:\mathbf{x}^i \in \mathcal{S}_k \cap \mathcal{R}_{st} } \dfdx(\mathbf{x}^i)
%\end{equation}
%
%\begin{equation}
%  \label{eq:bin_deviation_subregion}
%  \hat{\sigma}_{\mathcal{R}_{st}}^2(z_{k-1}, z_k) = \frac{1}{|\mathcal{S}_k \cap \mathcal{R}_{st}| - 1}
%\sum_{i:\mathbf{x}^i \in \mathcal{S}_k \cap \mathcal{R}_{st} } \left ( \dfdx(\mathbf{x}^i) -
%    \hat{\mu}(z_{k-1}, z_k) \right )^2
%\end{equation}
%
%\begin{equation}
%  \label{eq:SE}
%  \mathcal{H}_{\mathcal{R}_{sk}} = \sum_{k=1}^K (z_k - z_{k-1})^2 \hat{\sigma}_{\mathcal{R}_{st}}^2(z_{k-1}, z_k)
%\end{equation}

%\subsection{Algorithmic Details of Subregion Detection}
%
%\begin{algorithm}
%\caption{SplitDataset}
%\label{alg:split_dataset}
%\SetAlgoLined
%\SetKwInOut{Input}{Input}
%\SetKwInOut{Output}{Output}
%\Input{X\_list, J\_list, c, val, is\_cat}
%\Output{X\_split, J\_split}
%X\_split $\gets$ []\;
%J\_split $\gets$ []\;
%\For{i = 1 to len(X\_list)}{
%    X= X\_list[i]\;
%    J= J\_list[i]\;
%    \uIf{is\_cat}{
%        ind\_1 $\gets$ X[:, c] $=$ val\;
%        ind\_2 $\gets$ X[:, c] $\neq$ val\;
%    }
%    \Else{
%        ind\_1 $\gets$ X[:, c] $\leq$ val\;
%        ind\_2 $\gets$ X[:, c] $>$ val\;
%    }
%    Append X[ind\_1], X[ind\_2] to X\_split\;
%    Append J[ind\_1], J[ind\_2] to J\_split\;
%}
%\Return X\_split, J\_split
%\end{algorithm}


\bibliography{report}






\end{document}
